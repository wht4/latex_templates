%
% Generic Wireless Link - IMC Mittelbauprojekt
% ===========================================================================
% Autor: Wacher Tim
% 

\chapter{Einleitung}
\label{chap:einleitung}

% ===========================================================================
\section{Allgemeines}
\label{sec:einleitung_allgemeines}

\lipsum[1-3]


% ===========================================================================
\section{HTML-Code}
\label{sec:einleitung_html}

\begin{lstlisting}[style=LangXML, caption=Anfrage von jQuery an ein Servlet, 
label=lst:entscheidungen_ui_jquery]
<HTML>
    <HEAD>
        <script type="text/javascript" src="js/jquery-1.7.1.min.js"></script>
        <script type="text/javascript">
        $(document).ready(function(){
            $.ajax({  
                type: "GET",  
                url: "/servlet/test",
                success: function(data) {
                    alert(data);
                }
            });
        });
        </script>
    </HEAD>
    ...
\end{lstlisting}

% ===========================================================================
\section{JAVA-Code}
\label{sec:einleitung_java}

\begin{lstlisting}[style=LangJava, caption=Annotationen f�r die Klasse User, 
label=lst:entscheidungen_db_orm_user]
@Entity
@Table (name="USER_TABLE")
public class User {

    //----- Data ---------------------------------------------------------------
    @Id @GeneratedValue (strategy=GenerationType.AUTO)
    @Column(name="ID")
    private long id;
    
    @Column(name="NAME")
    private String name;
    
    @Column(name="INFORMATION")
    private String information;
    //----- Implementation -----------------------------------------------------
    
    // Getter and setter methods for all attributes
}
\end{lstlisting}

% ===========================================================================
\section{C-Code}
\label{sec:einleitung_C}

\begin{lstlisting}[style=LangC, caption=Kodieren einer JSon Nachricht mit Jansson, 
label=lst:entscheidungen_ifhosttarget_jsonnson]
json_t  psMsg = NULL;
char   *pcJsonText = NULL;

psMsg = json_pack("{s:s,s:s}", "KEY1", "VALUE1", "KEY2", "VALUE2");
pcJsonText = json_dumps(psMsg, JSON_COMPACT);
printf("%s\n", pcJsonText);
// Ausgabe: {"KEY1":"VALUE1","KEY2":"VALUE2"}
\end{lstlisting}

%
% ===========================================================================
% EOF
%
