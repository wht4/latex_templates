%% Buchvorlage unter Verwendung der Book-Klasse des KOMA-Script      %%
%% Basierend auf einer TeXNicCenter-Vorlage von Mark M�ller          %%
%%%%%%%%%%%%%%%%%%%%%%%%%%%%%%%%%%%%%%%%%%%%%%%%%%%%%%%%%%%%%%%%%%%%%%%

% W�hlen Sie die Optionen aus, indem Sie % vor der Option entfernen  
% Dokumentation des KOMA-Script-Packets: scrguide



%%%%%%%%%%%%%%%%%%%%%%%%%%%%%%%%%%%%%%%%%%%%%%%%%%%%%%%%%%%%%%%%%%%%%%%
%% Optionen zum Layout des Buchs                                     %%
%%%%%%%%%%%%%%%%%%%%%%%%%%%%%%%%%%%%%%%%%%%%%%%%%%%%%%%%%%%%%%%%%%%%%%%
\documentclass[
a4paper,                        % alle weiteren Papierformat einstellbar
%landscape,                     % Querformat
%10pt,                          % Schriftgre (12pt, 11pt (Standard))
%BCOR1cm,                       % Bindekorrektur, bspw. 1 cm
%DIVcalc,                       % f�hrt die Satzspiegelberechnung neu aus s. scrguide 2.4
%oneside,                       % einseitiges Layout
%twocolumn,                     % zweispaltiger Satz
openany,                        % Kapitel knnen auch auf linken Seiten beginnen
%halfparskip*,                  % Absatzformatierung s. scrguide 3.1
headsepline,                    % Trennline zum Seitenkopf    
footsepline,                    % Trennline zum Seitenfuss
%notitlepage,                   % in-page-Titel, keine eigene Titelseite
%chapterprefix,                  % vor Kapitelberschrift wird "Kapitel Nummer" gesetzt
appendixprefix,                 % Anhang wird "Anhang" vor die �berschrift gesetzt 
%normalheadings,                % berschriften etwas kleiner (smallheadings)
%idxtotoc,                       % Index im Inhaltsverzeichnis
%liststotoc,                    % Abb.- und Tab.verzeichnis im Inhalt
%bibtotoc,                      % Literaturverzeichnis im Inhalt
%leqno,                         % Nummerierung von Gleichungen links
%fleqn,                          % Ausgabe von Gleichungen linksbndig
%draft                          % �berlangen Zeilen in Ausgabe gekennzeichnet
]
{scrbook}
\usepackage{a4wide}




%%%%%%%%%%%%%%%%%%%%%%%%%%%%%%%%%%%%%%%%%%%%%%%%%%%%%%%%%%%%%%%%%%%%%%%
%% Deutsche Anpassung                                                %%
%%%%%%%%%%%%%%%%%%%%%%%%%%%%%%%%%%%%%%%%%%%%%%%%%%%%%%%%%%%%%%%%%%%%%%%
%\usepackage[ngerman]{babel}            % deutsch, neue Rechtschreibung
%\usepackage[T1]{fontenc}               % Silbentrennung bei Umlauten
%\usepackage[utf8]{inputenc}            % Zeichencodierung
\usepackage{german, ngerman}
\usepackage[german]{babel}

%Eingabe von �,�,�,� erlauben
% unter Linux:
\usepackage[latin1]{inputenc}
% unter Windows:
%\usepackage[ansinew]{inputenc}




%%%%%%%%%%%%%%%%%%%%%%%%%%%%%%%%%%%%%%%%%%%%%%%%%%%%%%%%%%%%%%%%%%%%%%%
%% Einstellungen                                                     %%
%%%%%%%%%%%%%%%%%%%%%%%%%%%%%%%%%%%%%%%%%%%%%%%%%%%%%%%%%%%%%%%%%%%%%%%

%\pagestyle{empty}              % keine Kopf und Fuzeile (k. Seitenzahl)
%\pagestyle{headings}           % lebender Kolumnentitel  




%%%%%%%%%%%%%%%%%%%%%%%%%%%%%%%%%%%%%%%%%%%%%%%%%%%%%%%%%%%%%%%%%%%%%%%
%% Packages                                                          %%
%%%%%%%%%%%%%%%%%%%%%%%%%%%%%%%%%%%%%%%%%%%%%%%%%%%%%%%%%%%%%%%%%%%%%%%

% Unterscheidung zw. pdf und dvi
%%%%%%%%%%%%%%%%%%%%%%%%%%%%%%%%%%%%%%%%%%%%%%%%%%%%%%%%%%%%%%%%%%%%%%
\usepackage{ifpdf}


% Text 1:1 �bernehmen
%%%%%%%%%%%%%%%%%%%%%%%%%%%%%%%%%%%%%%%%%%%%%%%%%%%%%%%%%%%%%%%%%%%%%%
\usepackage{verbatim} 


% Links im PDF
%%%%%%%%%%%%%%%%%%%%%%%%%%%%%%%%%%%%%%%%%%%%%%%%%%%%%%%%%%%%%%%%%%%%%%
\ifpdf
  \usepackage[
  pdftex,
  colorlinks,               % Schrift in Farbe, sonst mit Rahmen
  bookmarksnumbered,        % Inhaltsverzeichnis mit Numerierung
  bookmarksopen,            % �ffnet das Inhaltsverzeichnis
  %pdfstartview=FitH,       % startet mit Seitenbreite
  linkcolor=blue,          % standard red
  citecolor=blue,           % standard green
  urlcolor=blue,         % standard cyan
  filecolor=blue            %
  ]
  {hyperref}
\fi


% Farben
%%%%%%%%%%%%%%%%%%%%%%%%%%%%%%%%%%%%%%%%%%%%%%%%%%%%%%%%%%%%%%%%%%%%%%
\ifpdf
  \usepackage[pdftex]{color}
\else
  \usepackage[dvips]{color}
\fi

% Fonts f�r pdfLaTeX, falls keine cm-super-Fonts installiert
%%%%%%%%%%%%%%%%%%%%%%%%%%%%%%%%%%%%%%%%%%%%%%%%%%%%%%%%%%%%%%%%%%%%%%%
\ifpdf
    %\usepackage{ae}        % Benutzen Sie nur
    %\usepackage{zefonts}      % eines dieser Pakete
\else
    %%Normales LaTeX - keine speziellen Fontpackages notwendig
\fi


% Packages f�r Grafiken & Abbildungen
%%%%%%%%%%%%%%%%%%%%%%%%%%%%%%%%%%%%%%%%%%%%%%%%%%%%%%%%%%%%%%%%%%%%%%%
\ifpdf %%Einbindung von Grafiken mittels \includegraphics{datei}
    \usepackage[pdftex]{graphicx} %%Grafiken in pdfLaTeX
    \usepackage{epstopdf}
\else
    \usepackage[dvips]{graphicx} %%Grafiken und normales LaTeX
\fi
%\usepackage[hang]{subfigure} %%Mehrere Teilabbildungen in einer Abbildung
%\usepackage{pst-all} %%PSTricks - nicht verwendbar mit pdfLaTeX
\usepackage{wrapfig}
\usepackage{pdfpages}

% optischer Randausgleich, falls pdflatex
%%%%%%%%%%%%%%%%%%%%%%%%%%%%%%%%%%%%%%%%%%%%%%%%%%%%%%%%%%%%%%%%%%%%%%%
\ifpdf
  \usepackage[activate]{pdfcprot}
\fi


% Bibliographiestil
%%%%%%%%%%%%%%%%%%%%%%%%%%%%%%%%%%%%%%%%%%%%%%%%%%%%%%%%%%%%%%%%%%%%%%%
%\usepackage{natbib}


% Tabellen
%%%%%%%%%%%%%%%%%%%%%%%%%%%%%%%%%%%%%%%%%%%%%%%%%%%%%%%%%%%%%%%%%%%%%%%
\usepackage{booktabs}            % Tabellenlinien verschiedener Stärken
\usepackage{tabularx}            % fixe Gesamtbreite
\usepackage{threeparttable}      % Tabellen mit Fussnoten
\usepackage{multirow}            % Zeilen verbinden


% Diagramme
%%%%%%%%%%%%%%%%%%%%%%%%%%%%%%%%%%%%%%%%%%%%%%%%%%%%%%%%%%%%%%%%%%%%%%%
\usepackage{tikz}
\usetikzlibrary{shapes,arrows,mindmap,trees}


% Listings
%%%%%%%%%%%%%%%%%%%%%%%%%%%%%%%%%%%%%%%%%%%%%%%%%%%%%%%%%%%%%%%%%%%%%
\usepackage{xcolor}
\usepackage{colortbl}

\definecolor{Grey}{rgb}{0.5,0.5.0.5}
\definecolor{LightGrey}{rgb}{0.95,0.95,0.95}
\definecolor{Blue}{rgb}{0,0,1}
\definecolor{White}{rgb}{1,1,1}
\definecolor{Red}{rgb}{1,0,0}
\definecolor{Green}{rgb}{0,0.5,0}   
\definecolor{Violet}{rgb}{0.8,0,1}
\definecolor{grey}{rgb}{0.95,0.95,0.95}
\definecolor{blue}{rgb}{0,0,0.78}
\definecolor{red}{rgb}{1,0,0}
\definecolor{green}{rgb}{0,0.5,0}
\newcolumntype{g}{>{\columncolor{LightGrey}}p{3.5cm}}

\usepackage{listings}

\lstdefinestyle{LangC}{ 
 language=c,
 backgroundcolor=\color{LightGrey},
 keywordstyle=\color{Violet}\bfseries,
 commentstyle=\color{Green},
 stringstyle=\color{Blue},
 showstringspaces=false,
 basicstyle=\scriptsize,
 captionpos=b,
 tabsize=4,
 breaklines=true,
 numbers=none,   %none,left,...
 numberstyle=\tiny,
 numbersep=5pt
}

\lstdefinestyle{LangJava}{ 
 language=java,
 backgroundcolor=\color{LightGrey},
 keywordstyle=\color{Violet}\bfseries,
 commentstyle=\color{Green},
 stringstyle=\color{Blue},
 showstringspaces=false,
 basicstyle=\scriptsize,
 captionpos=b,
 tabsize=4,
 breaklines=true,
 numbers=none,   %none,left,...
 numberstyle=\tiny,
 numbersep=5pt
}

\lstdefinestyle{LangXML}{ 
 language=xml,
 backgroundcolor=\color{LightGrey},
 keywordstyle=\color{Violet}\bfseries,
 commentstyle=\color{Green},
 stringstyle=\color{Blue},
 showstringspaces=false,
 basicstyle=\scriptsize,
 captionpos=b,
 tabsize=4,
 breaklines=true,
 numbers=none,   %none,left,...
 numberstyle=\tiny,
 numbersep=5pt
}

\lstdefinestyle{LangHTML}{ 
 language=html,
 backgroundcolor=\color{LightGrey},
 keywordstyle=\color{Violet}\bfseries,
 commentstyle=\color{Green},
 stringstyle=\color{Blue},
 showstringspaces=false,
 basicstyle=\scriptsize,
 captionpos=b,
 tabsize=4,
 breaklines=true,
 numbers=none,   %none,left,...
 numberstyle=\tiny,
 numbersep=5pt
}


%%%%%%%%%%%%%%%%%%%%%%%%%%%%%%%%%%%%%%%%%%%%%%%%%%%%%%%%%%%%%%%%%%%%%%%
%% PDF-Infos                                                         %%
%%%%%%%%%%%%%%%%%%%%%%%%%%%%%%%%%%%%%%%%%%%%%%%%%%%%%%%%%%%%%%%%%%%%%%%
\ifpdf
  \pdfinfo{
  /Title (Generisches Testsystem)
  /Subject (MSc Engineering Master-Thesis)
  /Author (Wacher Tim)
  /Keywords (Generisches Testsystem, RTOS, Performance-Messungen, QNX)
  }
\fi




%%%%%%%%%%%%%%%%%%%%%%%%%%%%%%%%%%%%%%%%%%%%%%%%%%%%%%%%%%%%%%%%%%%%%%%
%% Dateiendungen f�r Grafiken                                        %%
%%%%%%%%%%%%%%%%%%%%%%%%%%%%%%%%%%%%%%%%%%%%%%%%%%%%%%%%%%%%%%%%%%%%%%%
\ifpdf
    \DeclareGraphicsExtensions{.pdf,.jpg,.png, .PDF, .JPG, .PNG}
\else
    \DeclareGraphicsExtensions{.eps}
\fi



%%%%%%%%%%%%%%%%%%%%%%%%%%%%%%%%%%%%%%%%%%%%%%%%%%%%%%%%%%%%%%%%%%%%%%%
%% Positionierung von Tabellen und Grafiken                          %%
%%%%%%%%%%%%%%%%%%%%%%%%%%%%%%%%%%%%%%%%%%%%%%%%%%%%%%%%%%%%%%%%%%%%%%%
\usepackage{float}
\usepackage{footnote}
\usepackage{lipsum}


%%%%%%%%%%%%%%%%%%%%%%%%%%%%%%%%%%%%%%%%%%%%%%%%%%%%%%%%%%%%%%%%%%%%%%%
%% Eigene Umgebungen und Befehle                                     %%
%%%%%%%%%%%%%%%%%%%%%%%%%%%%%%%%%%%%%%%%%%%%%%%%%%%%%%%%%%%%%%%%%%%%%%%

% Masseinheit in aufrechter Schrift und mit kleinem Zwischenraum zur Zahl
%%%%%%%%%%%%%%%%%%%%%%%%%%%%%%%%%%%%%%%%%%%%%%%%%%%%%%%%%%%%%%%%%%%%%%%
\newenvironment{unit}{\, \mathrm}{}

\definecolor{matlabcolor}{rgb}{0,0.5,0}
\newenvironment{matlab}{\texttt\bgroup\textcolor{matlabcolor}\bgroup }{ \egroup\egroup }
\newcommand{\ml}[1]{\texttt{\textcolor{matlabcolor}{#1}}}

%\renewcommand{\familydefault}{\sfdefault}
%\usepackage{helvet}

\usepackage{amsmath}
%\usepackage{amsfonts}
%\usepackage{amssymb} 
\usepackage{subfigure}
\setlength{\parindent}{0ex}

% Anhang mit Grossbuchstaben nummerieren
%%%%%%%%%%%%%%%%%%%%%%%%%%%%%%%%%%%%%%%%%%%%%%%%%%%%%%%%%%%%%%%%%%%%%%%
\newcommand{\initAnhang}{
	\renewcommand{\thepage}{\Alph{chapter}\ \arabic{page}}
	\newpage
}

\newcommand{\anhang}[1]{
    \setcounter{page}{1}
    \input{#1}
    \newpage
}

