%
% HW-Review Template 
% ===========================================================================
% Autor: Wacher Tim
% 

%% Buchvorlage unter Verwendung der Book-Klasse des KOMA-Script      %%
%% Basierend auf einer TeXNicCenter-Vorlage von Mark M�ller          %%
%%%%%%%%%%%%%%%%%%%%%%%%%%%%%%%%%%%%%%%%%%%%%%%%%%%%%%%%%%%%%%%%%%%%%%%

% W�hlen Sie die Optionen aus, indem Sie % vor der Option entfernen  
% Dokumentation des KOMA-Script-Packets: scrguide



%%%%%%%%%%%%%%%%%%%%%%%%%%%%%%%%%%%%%%%%%%%%%%%%%%%%%%%%%%%%%%%%%%%%%%%
%% Optionen zum Layout des Buchs                                     %%
%%%%%%%%%%%%%%%%%%%%%%%%%%%%%%%%%%%%%%%%%%%%%%%%%%%%%%%%%%%%%%%%%%%%%%%
\documentclass[
a4paper,                        % alle weiteren Papierformat einstellbar
%landscape,                     % Querformat
%10pt,                          % Schriftgre (12pt, 11pt (Standard))
%BCOR1cm,                       % Bindekorrektur, bspw. 1 cm
%DIVcalc,                       % f�hrt die Satzspiegelberechnung neu aus s. scrguide 2.4
%oneside,                       % einseitiges Layout
%twocolumn,                     % zweispaltiger Satz
openany,                        % Kapitel knnen auch auf linken Seiten beginnen
%halfparskip*,                  % Absatzformatierung s. scrguide 3.1
headsepline,                    % Trennline zum Seitenkopf    
footsepline,                    % Trennline zum Seitenfuss
%notitlepage,                   % in-page-Titel, keine eigene Titelseite
%chapterprefix,                  % vor Kapitelberschrift wird "Kapitel Nummer" gesetzt
appendixprefix,                 % Anhang wird "Anhang" vor die �berschrift gesetzt 
%normalheadings,                % berschriften etwas kleiner (smallheadings)
%idxtotoc,                       % Index im Inhaltsverzeichnis
%liststotoc,                    % Abb.- und Tab.verzeichnis im Inhalt
%bibtotoc,                      % Literaturverzeichnis im Inhalt
%leqno,                         % Nummerierung von Gleichungen links
%fleqn,                          % Ausgabe von Gleichungen linksbndig
%draft                          % �berlangen Zeilen in Ausgabe gekennzeichnet
]
{scrbook}
\usepackage{a4wide}




%%%%%%%%%%%%%%%%%%%%%%%%%%%%%%%%%%%%%%%%%%%%%%%%%%%%%%%%%%%%%%%%%%%%%%%
%% Deutsche Anpassung                                                %%
%%%%%%%%%%%%%%%%%%%%%%%%%%%%%%%%%%%%%%%%%%%%%%%%%%%%%%%%%%%%%%%%%%%%%%%
%\usepackage[ngerman]{babel}            % deutsch, neue Rechtschreibung
%\usepackage[T1]{fontenc}               % Silbentrennung bei Umlauten
%\usepackage[utf8]{inputenc}            % Zeichencodierung
\usepackage{german, ngerman}
\usepackage[german]{babel}

%Eingabe von �,�,�,� erlauben
% unter Linux:
\usepackage[latin1]{inputenc}
% unter Windows:
%\usepackage[ansinew]{inputenc}




%%%%%%%%%%%%%%%%%%%%%%%%%%%%%%%%%%%%%%%%%%%%%%%%%%%%%%%%%%%%%%%%%%%%%%%
%% Einstellungen                                                     %%
%%%%%%%%%%%%%%%%%%%%%%%%%%%%%%%%%%%%%%%%%%%%%%%%%%%%%%%%%%%%%%%%%%%%%%%

%\pagestyle{empty}              % keine Kopf und Fuzeile (k. Seitenzahl)
%\pagestyle{headings}           % lebender Kolumnentitel  




%%%%%%%%%%%%%%%%%%%%%%%%%%%%%%%%%%%%%%%%%%%%%%%%%%%%%%%%%%%%%%%%%%%%%%%
%% Packages                                                          %%
%%%%%%%%%%%%%%%%%%%%%%%%%%%%%%%%%%%%%%%%%%%%%%%%%%%%%%%%%%%%%%%%%%%%%%%

% Unterscheidung zw. pdf und dvi
%%%%%%%%%%%%%%%%%%%%%%%%%%%%%%%%%%%%%%%%%%%%%%%%%%%%%%%%%%%%%%%%%%%%%%
\usepackage{ifpdf}


% Text 1:1 �bernehmen
%%%%%%%%%%%%%%%%%%%%%%%%%%%%%%%%%%%%%%%%%%%%%%%%%%%%%%%%%%%%%%%%%%%%%%
\usepackage{verbatim} 


% Links im PDF
%%%%%%%%%%%%%%%%%%%%%%%%%%%%%%%%%%%%%%%%%%%%%%%%%%%%%%%%%%%%%%%%%%%%%%
\ifpdf
  \usepackage[
  pdftex,
  colorlinks,               % Schrift in Farbe, sonst mit Rahmen
  bookmarksnumbered,        % Inhaltsverzeichnis mit Numerierung
  bookmarksopen,            % �ffnet das Inhaltsverzeichnis
  %pdfstartview=FitH,       % startet mit Seitenbreite
  linkcolor=blue,          % standard red
  citecolor=blue,           % standard green
  urlcolor=blue,         % standard cyan
  filecolor=blue            %
  ]
  {hyperref}
\fi


% Farben
%%%%%%%%%%%%%%%%%%%%%%%%%%%%%%%%%%%%%%%%%%%%%%%%%%%%%%%%%%%%%%%%%%%%%%
\ifpdf
  \usepackage[pdftex]{color}
\else
  \usepackage[dvips]{color}
\fi

% Fonts f�r pdfLaTeX, falls keine cm-super-Fonts installiert
%%%%%%%%%%%%%%%%%%%%%%%%%%%%%%%%%%%%%%%%%%%%%%%%%%%%%%%%%%%%%%%%%%%%%%%
\ifpdf
    %\usepackage{ae}        % Benutzen Sie nur
    %\usepackage{zefonts}      % eines dieser Pakete
\else
    %%Normales LaTeX - keine speziellen Fontpackages notwendig
\fi


% Packages f�r Grafiken & Abbildungen
%%%%%%%%%%%%%%%%%%%%%%%%%%%%%%%%%%%%%%%%%%%%%%%%%%%%%%%%%%%%%%%%%%%%%%%
\ifpdf %%Einbindung von Grafiken mittels \includegraphics{datei}
    \usepackage[pdftex]{graphicx} %%Grafiken in pdfLaTeX
    \usepackage{epstopdf}
\else
    \usepackage[dvips]{graphicx} %%Grafiken und normales LaTeX
\fi
%\usepackage[hang]{subfigure} %%Mehrere Teilabbildungen in einer Abbildung
%\usepackage{pst-all} %%PSTricks - nicht verwendbar mit pdfLaTeX
\usepackage{wrapfig}
\usepackage{pdfpages}

% optischer Randausgleich, falls pdflatex
%%%%%%%%%%%%%%%%%%%%%%%%%%%%%%%%%%%%%%%%%%%%%%%%%%%%%%%%%%%%%%%%%%%%%%%
\ifpdf
  \usepackage[activate]{pdfcprot}
\fi


% Bibliographiestil
%%%%%%%%%%%%%%%%%%%%%%%%%%%%%%%%%%%%%%%%%%%%%%%%%%%%%%%%%%%%%%%%%%%%%%%
%\usepackage{natbib}


% Tabellen
%%%%%%%%%%%%%%%%%%%%%%%%%%%%%%%%%%%%%%%%%%%%%%%%%%%%%%%%%%%%%%%%%%%%%%%
\usepackage{booktabs}            % Tabellenlinien verschiedener Stärken
\usepackage{tabularx}            % fixe Gesamtbreite
\usepackage{threeparttable}      % Tabellen mit Fussnoten
\usepackage{multirow}            % Zeilen verbinden


% Diagramme
%%%%%%%%%%%%%%%%%%%%%%%%%%%%%%%%%%%%%%%%%%%%%%%%%%%%%%%%%%%%%%%%%%%%%%%
\usepackage{tikz}
\usetikzlibrary{shapes,arrows,mindmap,trees}


% Listings
%%%%%%%%%%%%%%%%%%%%%%%%%%%%%%%%%%%%%%%%%%%%%%%%%%%%%%%%%%%%%%%%%%%%%
\usepackage{xcolor}
\usepackage{colortbl}

\definecolor{Grey}{rgb}{0.5,0.5.0.5}
\definecolor{LightGrey}{rgb}{0.95,0.95,0.95}
\definecolor{Blue}{rgb}{0,0,1}
\definecolor{White}{rgb}{1,1,1}
\definecolor{Red}{rgb}{1,0,0}
\definecolor{Green}{rgb}{0,0.5,0}   
\definecolor{Violet}{rgb}{0.8,0,1}
\definecolor{grey}{rgb}{0.95,0.95,0.95}
\definecolor{blue}{rgb}{0,0,0.78}
\definecolor{red}{rgb}{1,0,0}
\definecolor{green}{rgb}{0,0.5,0}
\newcolumntype{g}{>{\columncolor{LightGrey}}p{3.5cm}}

\usepackage{listings}

\lstdefinestyle{LangC}{ 
 language=c,
 backgroundcolor=\color{LightGrey},
 keywordstyle=\color{Violet}\bfseries,
 commentstyle=\color{Green},
 stringstyle=\color{Blue},
 showstringspaces=false,
 basicstyle=\scriptsize,
 captionpos=b,
 tabsize=4,
 breaklines=true,
 numbers=none,   %none,left,...
 numberstyle=\tiny,
 numbersep=5pt
}

\lstdefinestyle{LangJava}{ 
 language=java,
 backgroundcolor=\color{LightGrey},
 keywordstyle=\color{Violet}\bfseries,
 commentstyle=\color{Green},
 stringstyle=\color{Blue},
 showstringspaces=false,
 basicstyle=\scriptsize,
 captionpos=b,
 tabsize=4,
 breaklines=true,
 numbers=none,   %none,left,...
 numberstyle=\tiny,
 numbersep=5pt
}

\lstdefinestyle{LangXML}{ 
 language=xml,
 backgroundcolor=\color{LightGrey},
 keywordstyle=\color{Violet}\bfseries,
 commentstyle=\color{Green},
 stringstyle=\color{Blue},
 showstringspaces=false,
 basicstyle=\scriptsize,
 captionpos=b,
 tabsize=4,
 breaklines=true,
 numbers=none,   %none,left,...
 numberstyle=\tiny,
 numbersep=5pt
}

\lstdefinestyle{LangHTML}{ 
 language=html,
 backgroundcolor=\color{LightGrey},
 keywordstyle=\color{Violet}\bfseries,
 commentstyle=\color{Green},
 stringstyle=\color{Blue},
 showstringspaces=false,
 basicstyle=\scriptsize,
 captionpos=b,
 tabsize=4,
 breaklines=true,
 numbers=none,   %none,left,...
 numberstyle=\tiny,
 numbersep=5pt
}


%%%%%%%%%%%%%%%%%%%%%%%%%%%%%%%%%%%%%%%%%%%%%%%%%%%%%%%%%%%%%%%%%%%%%%%
%% PDF-Infos                                                         %%
%%%%%%%%%%%%%%%%%%%%%%%%%%%%%%%%%%%%%%%%%%%%%%%%%%%%%%%%%%%%%%%%%%%%%%%
\ifpdf
  \pdfinfo{
  /Title (Generisches Testsystem)
  /Subject (MSc Engineering Master-Thesis)
  /Author (Wacher Tim)
  /Keywords (Generisches Testsystem, RTOS, Performance-Messungen, QNX)
  }
\fi




%%%%%%%%%%%%%%%%%%%%%%%%%%%%%%%%%%%%%%%%%%%%%%%%%%%%%%%%%%%%%%%%%%%%%%%
%% Dateiendungen f�r Grafiken                                        %%
%%%%%%%%%%%%%%%%%%%%%%%%%%%%%%%%%%%%%%%%%%%%%%%%%%%%%%%%%%%%%%%%%%%%%%%
\ifpdf
    \DeclareGraphicsExtensions{.pdf,.jpg,.png, .PDF, .JPG, .PNG}
\else
    \DeclareGraphicsExtensions{.eps}
\fi



%%%%%%%%%%%%%%%%%%%%%%%%%%%%%%%%%%%%%%%%%%%%%%%%%%%%%%%%%%%%%%%%%%%%%%%
%% Positionierung von Tabellen und Grafiken                          %%
%%%%%%%%%%%%%%%%%%%%%%%%%%%%%%%%%%%%%%%%%%%%%%%%%%%%%%%%%%%%%%%%%%%%%%%
\usepackage{float}
\usepackage{footnote}
\usepackage{lipsum}


%%%%%%%%%%%%%%%%%%%%%%%%%%%%%%%%%%%%%%%%%%%%%%%%%%%%%%%%%%%%%%%%%%%%%%%
%% Eigene Umgebungen und Befehle                                     %%
%%%%%%%%%%%%%%%%%%%%%%%%%%%%%%%%%%%%%%%%%%%%%%%%%%%%%%%%%%%%%%%%%%%%%%%

% Masseinheit in aufrechter Schrift und mit kleinem Zwischenraum zur Zahl
%%%%%%%%%%%%%%%%%%%%%%%%%%%%%%%%%%%%%%%%%%%%%%%%%%%%%%%%%%%%%%%%%%%%%%%
\newenvironment{unit}{\, \mathrm}{}

\definecolor{matlabcolor}{rgb}{0,0.5,0}
\newenvironment{matlab}{\texttt\bgroup\textcolor{matlabcolor}\bgroup }{ \egroup\egroup }
\newcommand{\ml}[1]{\texttt{\textcolor{matlabcolor}{#1}}}

%\renewcommand{\familydefault}{\sfdefault}
%\usepackage{helvet}

\usepackage{amsmath}
%\usepackage{amsfonts}
%\usepackage{amssymb} 
\usepackage{subfigure}
\setlength{\parindent}{0ex}

% Anhang mit Grossbuchstaben nummerieren
%%%%%%%%%%%%%%%%%%%%%%%%%%%%%%%%%%%%%%%%%%%%%%%%%%%%%%%%%%%%%%%%%%%%%%%
\newcommand{\initAnhang}{
	\renewcommand{\thepage}{\Alph{chapter}\ \arabic{page}}
	\newpage
}

\newcommand{\anhang}[1]{
    \setcounter{page}{1}
    \input{#1}
    \newpage
}


\title{Software Review}
\author{Wacher Tim}


\begin{document}
\ \\[0cm]
\begin{tikzpicture}
\draw(0,0)--(\textwidth,0)--(\textwidth,-2cm)--(0,-2cm)--(0,0);
\node at (0.32\textwidth,-1){%
\begin{minipage}[t]{0.6\textwidth}
\huge{\textsf{Software Review}}\\[0.5ex]
\large{\textsf{Projektname:}}\\[0.6ex]
\large{\textsf{Modul:}}
\end{minipage}
};
\node at (\textwidth+3mm,-1.25){%
\begin{minipage}[t]{0.8\textwidth}
\textsf{Version:}\\[1.8mm]
\textsf{Reviewer:}\\[1.8mm]
\textsf{Datum:}\\
\end{minipage}
};


\AddToShipoutPicture{%
\put(80,40){%
\begin{tikzpicture}
\draw(0,0.8cm)--(\textwidth,0.8);
\node at(140mm,4mm)[anchor=west]{\textsf{page \arabic{page}}};
\end{tikzpicture}
}}


\AddToShipoutPicture{%
\put(80,760){%
\begin{tikzpicture}
\draw(0,-0.8cm)--(\textwidth,-0.8);
\node[inner sep=0pt] (russell) at (1,0)
    {\includegraphics[width=.075\textwidth]{img/icon.png}};
\end{tikzpicture}
}}

\draw (0.6\textwidth,0)--(0.6\textwidth,-2cm);
\draw (0.6\textwidth,-0.666cm)--(\textwidth,-0.666cm);
\draw (0.6\textwidth,-1.333cm)--(\textwidth,-1.333cm);
\draw (0.77\textwidth,0)--(0.77\textwidth,-2cm);
\end{tikzpicture}


% ===========================================================================
\section{SW-Review Zusammenfassung}

\textbf{Notwendige �nderungen:} \newline
\kommentarfeld[23] 

% ===========================================================================
\section{Allgemein}

\requirement[2.1 (advisory), {Ist der Header der Files und jeder Funktion beschreibend 
genug}, Approved]

\requirement[2.2 (advisory), {Jedes File kompiliert alleine ohne Fehler/Warnings (auch Header-Files)}, Approved]

\requirement[2.3 (advisory), {Revision History enth�lt alle �nderungen}, Approved]

\requirement[2.4 (advisory), {Gibt es unerreichbare Codeabschnitte (auskommentierter Code 
oder nicht erreichbarer Code) welche entfernt werden sollten}, Approved]

\requirement[2.5 (advisory), {Musste der Autor gefragt werden was der Code macht (Code 
sollte selbsterkl�rend sein)}, Approved]

% ===========================================================================
\section{Comments}

\requirement[3.1 (advisory), {Entspricht der Kommentar dem Code}, Approved]

\requirement[3.2 (advisory), {Jede Funktion beschreibt die notwendigen Paramter (vor 
allem jene welche ver�ndert werden) und m�gliche Funktions-Abh�ngigkeiten}, Approved]

\requirement[3.3 (advisory), {Kein �berfl�ssiger Kommentar \newline 
(z.B.\,\, \textit{i++; \,\,\,// increment i)})}, Approved]

% ===========================================================================
\section{Coding Standards}

\requirement[4.1 (advisory), {Konstanten und Literale sind nicht hard coded}, Approved]

\requirement[4.2 (advisory), {Debug-Fehlermeldungen sind verst�ndlich und komplett}, Approved]

\requirement[4.3 (advisory), {Die Struktur des Codes wird durch Einr�cken verdeutlicht (keine Hard-Tabs)}, Approved]

\requirement[4.4 (advisory), {Klammern schaffen Klarheit und werden grossz�gig eingesetzt}, Approved]

\requirement[4.5 (advisory), {Der Code-Style ist innerhalb des Moduls konsistent und gut strukturiert}, Approved]

\requirement[4.6 (advisory), {Assembler-Code wird geeignet gekapselt (innerhalb Macro/Funktion 
oder seperatem Assembler-File)}, Approved]

\requirement[4.7 (advisory), {Es gibt keine auskommentierten Codeabschnitte (es werden 
\#if or \#ifdef Preprocessorbefehle eingesetzt)}, Approved]

\requirement[4.8 (advisory), {goto oder continue Befehle werden nicht verwendet}, Approved]

\requirement[4.9 (advisory), {Makros mit Argumenten wurden durch inline Funktionen ersetzt}, Approved]

% ===========================================================================
\section{Kontrollstrukturen}

\requirement[5.1 (required), {Korrekte Klammersetzung bei Bl�cken von Schlaufen 
und Verzweigungen}, Approved]

\requirement[5.2 (required), {Schlaufen: \begin{itemize}
\item Endbedingung von Schlaufen �berpr�ft
\item Schalufen-Z�hler werden korrekt initialisiert (vor der Schlaufe)
\item Alle Schlaufen-Variablen werden vor der Schlaufe initialisiert
\end{itemize}}, Approved]

\requirement[5.3 (required), {Switch Verzweigung: \begin{itemize}
\item Gibt es einen default case
\item Jeder non-empty switch case wird mit einem break abgeschlossen
\end{itemize}}, Approved]

\requirement[5.4 (required), {if ... else if  Verzweigung: \begin{itemize}
\item Werden die meist eintreffenden F�lle zuerst getestet
\item Wird der else Fall behandelt
\end{itemize}}, Approved]

\requirement[5.5 (advisory), {Der Geltungsbereich von Kontrollstrukturen (switch, while, 
do ... while, for, if, else if und else ) wird durch geschweifte Klammern verdeutlicht}, Approved]


% ===========================================================================
\section{Variablen}

\requirement[6.1 (required), {Alle Variablen werden vor der Verwendung initialiert}, Approved]

\requirement[6.2 (required), {Haben alle Variablen den korrekten Typ oder Cast}, Approved]

\requirement[6.3 (required), {Geltungsbereich von Variablen: \begin{itemize}
\item Wurden globale Variablen minimal eingesetzt
\item Haben alle Variablen den kleinst m�glichen Geltungsbereich
\end{itemize}}, Approved]

\requirement[6.4 (advisory), {Werden globale Variablen korrekt initialisiert (behalten diese 
bei einem Reset den notwendigen Wert)}, Approved]

\requirement[6.5 (advisory), {Gibt es redundante oder unbenutzte Variablen}, Approved]

\requirement[6.6 (advisory), {Alle Variablen haben einen klaren und beschreibenden Namen}, Approved]

% ===========================================================================
\section{Datentypen}

\requirement[7.1 (required), {Verwendung von portablen Datentypen (z.B. uint32\_t)}, Approved]

\requirement[7.2 (required), {Wurde das Vorzeichen des Datentyps bei der Deklaration 
spezifiziert (Verwendung von unsigned/signed)}, Approved]

\requirement[7.3 (required), {Wurden Datentypen mit typedef deklariert}, Approved]


% ===========================================================================
\section{volatile}

\requirement[9.1 (required), {Alle Memmory-Mapped Peripherie Register sind als volatile 
deklariert}, Approved]

\requirement[9.2 (required), {Globale Variablen welche in ISR verwendet werden sind als 
volatile deklariert}, Approved]

\requirement[9.3 (required), {Globale Variablen welche aus unterschiedlichen Tasks verwendet 
werden sind als volatile deklariert (auch wenn durch Mutex/Semaphore gesch�tzt)}, Approved]

% ===========================================================================
\section{Array}

\requirement[10.1 (required), {Array Indizes sind innerhalb der definierten Grenzen}, Approved]

\requirement[10.2 (required), {Wird das Verlassen der Bereichsgrenzen von Array-Indizes 
�berpr�ft (dies gilt auch f�r Pointer)}, Approved]

\requirement[10.3 (required), {Auf $\mu$Controller keine grossen lokalen Arrays (Stackoverflow)}, Approved]

% ===========================================================================
\section{Logic/Arithmetic}

\requirement[11.1 (required), {Division durch Null abgefangen/�berpr�ft}, Approved]

\requirement[11.2 (required), {Kein Vergleich von Floating-Point Zahlen auf Gleichheit}, Approved]

\requirement[11.3 (required), {Wird dynamisch allozierter Speicher wieder korrekt freigegeben 
(nur allozierter Speicher freigeben)}, Approved]

\requirement[11.4 (required), {Der Zugriff auf externe Devices wird durch Timeouts gesch�tzt}, Approved]

\requirement[11.5 (required), {Gibt es eine Chance auf einen mathematischen Overflow/Underflow 
(ist dies abgefangen)}, Approved]

\requirement[11.6 (required), {Ist der Einstatz von Bitweise, Relationale und Logischen 
Operatoren korrekt (korrekter Einsatz von ==, =, \&\&, \&, etc.; 
korrekt geklammert (a \& 0x01) == 0 $\rightarrow$ Gleich bindet mehr als \&)}, Approved]

% ===========================================================================
\section{Devensive Programmierweise}

\requirement[12.1 (required), {Werden Paramter beim Funktionseintritt auf ihre 
G�ltigkeit �berpr�ft (sanity checking)}, Approved]

\requirement[12.2 (required), {R�ckgabewerte von Funktionen (vor allem Fehlermeldungen) 
werden �berpr�ft}, Approved]

\requirement[12.3 (required), {Wird auf m�gliche NULL-Pointer hin getestet}, Approved]


% ===========================================================================
\section{Funktionen}

\requirement[13.1 (required), {Werden alle lokalen Funktionsvariablen zu Beginn initialisiert}, Approved]

\requirement[13.2 (required), {Werden Fehlerwerte generiert und der aufrufenden 
Funktion zur�ckgegeben}, Approved]

\requirement[13.3 (required), {Ist die Funktion im minimalen Geltungsbereich definiert}, Approved]

\requirement[13.4 (advisory), {Alle Funktionen haben einen klaren und beschreibenden Namen}, Approved]

\requirement[13.5 (advisory), {Gibt es nicht verwendete Funktionen welche gel�scht werden k�nnten}, Approved]

\requirement[13.6 (advisory), {Sind Funktionen nicht zu komplex (allenfalls Aufteilung in mehrere 
Funktionen; Richtwert: ausgedruckt max. 1 A4-Seite)}, Approved]

\requirement[13.7 (advisory), {Gibt es redundanten Code welcher in einer Funktion gekapselt werden kann}, Approved]

\requirement[13.8 (advisory), {Jede Funktion hat nur einen Exit-Punkt (keine return Statement in der 
Mitte der Funktion)}, Approved]

% ===========================================================================
\section{Multithreading}

\requirement[14.1 (required), {Sind Task-Priorit�ten sinnvoll gew�hlt}, Approved]

\requirement[14.2 (required), {Kontrolle der Stackauslastung der einzelnen Tasks}, Approved]

\requirement[14.3 (required), {Semaphoren/Mutex: \begin{itemize}
\item Zugriff auf kritische Abschnitte ist gesch�tzt
\item Jede angeforderte Semaphore/Mutex wird wieder zur�ckgegeben
\item Alle Semaphoren/Mutex werden in der selben Reihenfolge angefordert und 
in umgekehrter Reihenfolge zur�ckgegeben (Vermeidung von Deadlocks) 
\item Zum Schutz von kritischen Abschnitten werden Mutexe verwendet (Vermeidung 
von Priority Inversion)
\item Es wird kein Task gel�scht welcher eine Semaphore/Mutex belegt
\item Rekursive Mutexe werden korrekt eingesetzt (Verschachtelung)
\end{itemize}}, Approved]

\requirement[14.4 (required), {Es werden keine non-reentrant Funktionen aus den Tasks 
aufgerufen (Vorsicht bei Library-Funktionen)}, Approved]

% ===========================================================================
\section{Interrupt Service Routine}

\requirement[15.1 (required), {Innerhalb von ISR werden nur geeignete RTOS-API Funktionen 
aufgerufen}, Approved]

\requirement[15.2 (required), {Es werden keine non-reentrant Funktionen aus der ISR 
aufgerufen (Vorsicht bei Library-Funktionen)}, Approved]

\requirement[15.3 (advisory), {ISR ist so kurz als m�glich gehalten}, Approved]

% ===========================================================================
\section{$\mu$Controller-Peripherie}

\requirement[16.1 (required), {Watchdog: \begin{itemize}
\item Sinnvolles Timeout gew�hlt
\item Reset des Watchdogs an geeigneten Orten (nicht zu tief verschachtelt)
\item  Initialisierungsvorgang durch Watchdog gesch�tzt
\item Spricht der Watchdog an (Test durch forcieren eines Watchdog-Resets 
an einigen Code-Stellen)
\end{itemize}}, Approved]



% ===========================================================================
\end{document}
%
% ===========================================================================
% EOF
%
