%%%%%%%%%%%%%%%%%%%%%%%%%%%%%%%%%%%%%%%%%%%%%%%%%%%%%%%%%%%%%%%%%%%%%%%
%% Optionen zum Layout des Reviews                                   %%
%%%%%%%%%%%%%%%%%%%%%%%%%%%%%%%%%%%%%%%%%%%%%%%%%%%%%%%%%%%%%%%%%%%%%%%
\documentclass[
a4paper,                        % alle weiteren Papierformat einstellbar
%landscape,                     % Querformat
%10pt,                          % Schriftgre (12pt, 11pt (Standard))
%BCOR1cm,                       % Bindekorrektur, bspw. 1 cm
%DIVcalc,                       % führt die Satzspiegelberechnung neu aus s. scrguide 2.4
%oneside,                       % einseitiges Layout
%twocolumn,                     % zweispaltiger Satz
%openany,                        % Kapitel knnen auch auf linken Seiten beginnen
%halfparskip*,                  % Absatzformatierung s. scrguide 3.1
%headsepline,                    % Trennline zum Seitenkopf    
%footsepline,                    % Trennline zum Seitenfuss
notitlepage,                   % in-page-Titel, keine eigene Titelseite
%chapterprefix,                  % vor Kapitelberschrift wird "Kapitel Nummer" gesetzt
%appendixprefix,                 % Anhang wird "Anhang" vor die Überschrift gesetzt 
%normalheadings,                % berschriften etwas kleiner (smallheadings)
%idxtotoc,                       % Index im Inhaltsverzeichnis
%liststotoc,                    % Abb.- und Tab.verzeichnis im Inhalt
%bibtotoc,                      % Literaturverzeichnis im Inhalt
%leqno,                         % Nummerierung von Gleichungen links
%fleqn,                          % Ausgabe von Gleichungen linksbndig
%draft                          % Überlangen Zeilen in Ausgabe gekennzeichnet
]
{scrartcl}
\usepackage{a4wide}



%%%%%%%%%%%%%%%%%%%%%%%%%%%%%%%%%%%%%%%%%%%%%%%%%%%%%%%%%%%%%%%%%%%%%%%
%% Deutsche Anpassung                                                %%
%%%%%%%%%%%%%%%%%%%%%%%%%%%%%%%%%%%%%%%%%%%%%%%%%%%%%%%%%%%%%%%%%%%%%%%
\usepackage{german, ngerman}
\usepackage[german]{babel}
%Eingabe von ü,ä,ö,ß erlauben
% unter Linux:
\usepackage[latin1]{inputenc}

%%%%%%%%%%%%%%%%%%%%%%%%%%%%%%%%%%%%%%%%%%%%%%%%%%%%%%%%%%%%%%%%%%%%%%%
%% Packages                                                          %%
%%%%%%%%%%%%%%%%%%%%%%%%%%%%%%%%%%%%%%%%%%%%%%%%%%%%%%%%%%%%%%%%%%%%%%%

% Unterscheidung zw. pdf und dvi
%%%%%%%%%%%%%%%%%%%%%%%%%%%%%%%%%%%%%%%%%%%%%%%%%%%%%%%%%%%%%%%%%%%%%%
\usepackage{ifpdf}

% Text 1:1 Übernehmen
%%%%%%%%%%%%%%%%%%%%%%%%%%%%%%%%%%%%%%%%%%%%%%%%%%%%%%%%%%%%%%%%%%%%%%
\usepackage{verbatim} 

% Links im PDF
%%%%%%%%%%%%%%%%%%%%%%%%%%%%%%%%%%%%%%%%%%%%%%%%%%%%%%%%%%%%%%%%%%%%%%
\ifpdf
  \usepackage[
  pdftex,
  colorlinks,               % Schrift in Farbe, sonst mit Rahmen
  bookmarksnumbered,        % Inhaltsverzeichnis mit Numerierung
  bookmarksopen,            % Öffnet das Inhaltsverzeichnis
  %pdfstartview=FitH,       % startet mit Seitenbreite
  linkcolor=blue,          % standard red
  citecolor=blue,           % standard green
  urlcolor=blue,         % standard cyan
  filecolor=blue            %
  ]
  {hyperref}
\fi

% Farben
%%%%%%%%%%%%%%%%%%%%%%%%%%%%%%%%%%%%%%%%%%%%%%%%%%%%%%%%%%%%%%%%%%%%%%
\ifpdf
  \usepackage[pdftex]{color}
\else
  \usepackage[dvips]{color}
\fi 

% Packages für Grafiken & Abbildungen
%%%%%%%%%%%%%%%%%%%%%%%%%%%%%%%%%%%%%%%%%%%%%%%%%%%%%%%%%%%%%%%%%%%%%%%
\ifpdf %%Einbindung von Grafiken mittels \includegraphics{datei}
    \usepackage[pdftex]{graphicx} %%Grafiken in pdfLaTeX
    \usepackage{epstopdf}
\else
    \usepackage[dvips]{graphicx} %%Grafiken und normales LaTeX
\fi 

\usepackage{wrapfig}
\usepackage{pdfpages}


% optischer Randausgleich, falls pdflatex
%%%%%%%%%%%%%%%%%%%%%%%%%%%%%%%%%%%%%%%%%%%%%%%%%%%%%%%%%%%%%%%%%%%%%%%
\ifpdf
  \usepackage[activate]{pdfcprot}
\fi
 
% Tabellen
%%%%%%%%%%%%%%%%%%%%%%%%%%%%%%%%%%%%%%%%%%%%%%%%%%%%%%%%%%%%%%%%%%%%%%%
\usepackage{booktabs}            % Tabellenlinien verschiedener Stärken
\usepackage{tabularx}            % fixe Gesamtbreite
\usepackage{threeparttable}      % Tabellen mit Fussnoten
\usepackage{multirow}            % Zeilen verbinden

% Diagramme
%%%%%%%%%%%%%%%%%%%%%%%%%%%%%%%%%%%%%%%%%%%%%%%%%%%%%%%%%%%%%%%%%%%%%%%
\usepackage{tikz}
\usetikzlibrary{shapes,arrows,mindmap,trees}

% Listings
%%%%%%%%%%%%%%%%%%%%%%%%%%%%%%%%%%%%%%%%%%%%%%%%%%%%%%%%%%%%%%%%%%%%%
\usepackage{xcolor}
\usepackage{colortbl}

\definecolor{Grey}{rgb}{0.5,0.5.0.5}
\definecolor{LightGrey}{rgb}{0.95,0.95,0.95}
\definecolor{Blue}{rgb}{0,0,1}
\definecolor{White}{rgb}{1,1,1}
\definecolor{Red}{rgb}{1,0,0}
\definecolor{Green}{rgb}{0,0.5,0}   
\definecolor{Violet}{rgb}{0.8,0,1}
\definecolor{grey}{rgb}{0.95,0.95,0.95}
\definecolor{blue}{rgb}{0,0,0.78}
\definecolor{red}{rgb}{1,0,0}
\definecolor{green}{rgb}{0,0.5,0}
\newcolumntype{g}{>{\columncolor{LightGrey}}p{3.5cm}}

\usepackage{listings}

\lstdefinestyle{LangC}{ 
 language=c,
 backgroundcolor=\color{LightGrey},
 keywordstyle=\color{Violet}\bfseries,
 commentstyle=\color{Green},
 stringstyle=\color{Blue},
 showstringspaces=false,
 basicstyle=\scriptsize,
 captionpos=b,
 tabsize=4,
 breaklines=true,
 numbers=none,   %none,left,...
 numberstyle=\tiny,
 numbersep=5pt
}

\lstdefinestyle{LangJava}{ 
 language=java,
 backgroundcolor=\color{LightGrey},
 keywordstyle=\color{Violet}\bfseries,
 commentstyle=\color{Green},
 stringstyle=\color{Blue},
 showstringspaces=false,
 basicstyle=\scriptsize,
 captionpos=b,
 tabsize=4,
 breaklines=true,
 numbers=none,   %none,left,...
 numberstyle=\tiny,
 numbersep=5pt
}

\lstdefinestyle{LangXML}{ 
 language=xml,
 backgroundcolor=\color{LightGrey},
 keywordstyle=\color{Violet}\bfseries,
 commentstyle=\color{Green},
 stringstyle=\color{Blue},
 showstringspaces=false,
 basicstyle=\scriptsize,
 captionpos=b,
 tabsize=4,
 breaklines=true,
 numbers=none,   %none,left,...
 numberstyle=\tiny,
 numbersep=5pt
}

\lstdefinestyle{LangHTML}{ 
 language=html,
 backgroundcolor=\color{LightGrey},
 keywordstyle=\color{Violet}\bfseries,
 commentstyle=\color{Green},
 stringstyle=\color{Blue},
 showstringspaces=false,
 basicstyle=\scriptsize,
 captionpos=b,
 tabsize=4,
 breaklines=true,
 numbers=none,   %none,left,...
 numberstyle=\tiny,
 numbersep=5pt
}


%%%%%%%%%%%%%%%%%%%%%%%%%%%%%%%%%%%%%%%%%%%%%%%%%%%%%%%%%%%%%%%%%%%%%%%
%% PDF-Infos                                                         %%
%%%%%%%%%%%%%%%%%%%%%%%%%%%%%%%%%%%%%%%%%%%%%%%%%%%%%%%%%%%%%%%%%%%%%%%
\ifpdf
  \pdfinfo{
  /Title (Generisches Testsystem)
  /Subject (MSc Engineering Master-Thesis)
  /Author (Wacher Tim)
  /Keywords (Generisches Testsystem, RTOS, Performance-Messungen, QNX)
  }
\fi




%%%%%%%%%%%%%%%%%%%%%%%%%%%%%%%%%%%%%%%%%%%%%%%%%%%%%%%%%%%%%%%%%%%%%%%
%% Dateiendungen für Grafiken                                        %%
%%%%%%%%%%%%%%%%%%%%%%%%%%%%%%%%%%%%%%%%%%%%%%%%%%%%%%%%%%%%%%%%%%%%%%%
\ifpdf
    \DeclareGraphicsExtensions{.pdf,.jpg,.png, .PDF, .JPG, .PNG}
\else
    \DeclareGraphicsExtensions{.eps}
\fi



%%%%%%%%%%%%%%%%%%%%%%%%%%%%%%%%%%%%%%%%%%%%%%%%%%%%%%%%%%%%%%%%%%%%%%%
%% Positionierung von Tabellen und Grafiken                          %%
%%%%%%%%%%%%%%%%%%%%%%%%%%%%%%%%%%%%%%%%%%%%%%%%%%%%%%%%%%%%%%%%%%%%%%%
\usepackage{float}
\usepackage{footnote}



%%%%%%%%%%%%%%%%%%%%%%%%%%%%%%%%%%%%%%%%%%%%%%%%%%%%%%%%%%%%%%%%%%%%%%%
%% Eigene Umgebungen und Befehle                                     %%
%%%%%%%%%%%%%%%%%%%%%%%%%%%%%%%%%%%%%%%%%%%%%%%%%%%%%%%%%%%%%%%%%%%%%%%

% Masseinheit in aufrechter Schrift und mit kleinem Zwischenraum zur Zahl
%%%%%%%%%%%%%%%%%%%%%%%%%%%%%%%%%%%%%%%%%%%%%%%%%%%%%%%%%%%%%%%%%%%%%%%
\newenvironment{unit}{\, \mathrm}{}

\definecolor{matlabcolor}{rgb}{0,0.5,0}
\newenvironment{matlab}{\texttt\bgroup\textcolor{matlabcolor}\bgroup }{ \egroup\egroup }
\newcommand{\ml}[1]{\texttt{\textcolor{matlabcolor}{#1}}}

%\renewcommand{\familydefault}{\sfdefault}
%\usepackage{helvet}

\usepackage{amsmath}
%\usepackage{amsfonts}
%\usepackage{amssymb} 
\usepackage{subfigure}
\setlength{\parindent}{0ex}

\usepackage[T1]{fontenc}
\usepackage[scaled]{helvet}
\usepackage{mathpazo}

%\renewcommand{\familydefault}{\sfdefault}

\pagestyle{empty}


%%%%%%%%%%%%%%%%%%%%%%%%%%%%%%%%%%%%%%%
% Eigene Konstrukte
%%%%%%%%%%%%%%%%%%%%%%%%%%%%%%%%%%%%%%%

\def\us{\char`\_}

\def\tickbox[#1]{
%\noindent
\begin{tikzpicture}
\begin{scope}%[node distance=18mm]
  \node at (0,0) [right] {#1};
  
  \node at (2,0) [right, draw, minimum height=4mm, minimum width=4mm] {};
\end{scope}
\end{tikzpicture}
\ \\[4ex]
}

\def\requirement[#1, #2, #3]{
\begin{table}[H]
	\centering
	\small
	\begin{tabular}{p{1cm} p{9.5cm} p{1cm} p{5cm}}
        \textbf{#1} & {#2} & & \vspace{-4mm} \tickbox[#3] \\
    \end{tabular}
\end{table}
}

\def\kommentarfeld[#1]{%
\begin{tikzpicture}
\draw[step=4mm,color=black,dotted](0,0)grid(\textwidth,#1*4mm);
\end{tikzpicture}
}

\def\messeinstellung[#1]{%
\begin{tikzpicture}
\draw [step=4mm,color=black,dotted](5*4mm,0)grid(\textwidth,#1*4mm);
\end{tikzpicture}
}


\def\messung[#1]{%
\begin{tikzpicture}[node distance=3cm]
\node (#1:mes)[minimum width=2cm]{#1};
\node (#1:val) [draw,minimum height=8mm,minimum width=4cm,anchor=west,right of=#1:mes] {};

\end{tikzpicture}
\ \\[1ex]
}
